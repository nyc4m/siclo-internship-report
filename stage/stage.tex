\documentclass[../rapport.tex]{subfiles}

\begin{document}
    \chapter{Le stage}
    \section{Généralité}
    Dans le but de développer des applications web modernes et évolutives, 
    la tendance actuellement est de découper un projet en deux parties : une
    partie côté client, et une partie côté serveur.

    La partie côté client correspond
    alors à ce qui est affiché à l'utilisateur, le développeur aura donc la responsabilité
    de mettre en forme les données, mais aussi de développer une interface agréable à utiliser
    afin que l'utilisateur puisse utiliser l'application de la meilleure maniere possible.
    On parle dans ce cas de créer une \og Expérience utilisateur\fg~

    En parallèle, la partie serveur est responsable du traitement des données. On entend par traitement
    des données, la manière dont elles sont stockées, mais aussi comment est régulé leur accès.
    Lorsque l'on s'occupe du developpement côté serveur il convient donc de définir les données que l'on
    va utiliser, mais aussi de définir un ensemble d'action possible pour pouvoir créer, lire, mettre à jour
    mais aussi supprimer les données de l'application développée.

    \section{Role}
     Mon stage a été l'occasion pour moi de faire du développement côté serveur. Dans ce role j'ai eu l'occasion de
     participer à trois projets. Travailler sur plusieurs projets m'a ainsi permis de me former ou m'ameliorer sur php et nodejs
     mais aussi d'apprendre à utiliser un message broker avec le protocol AMQ ou encore de comprendre comment fonctionnent les notifications
     push.

    \end{document}
